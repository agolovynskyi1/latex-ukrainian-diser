\chapter{ВСТУП}

Ресурсний центр УНГ на основі суперкомп’ютерного комплексу СКІТ є найбільшим обчислюівальним центром в Україні, що надає науковцям НАН України можливості для виконання складних ресурсоємних науково-технічних розрахунків, і є однією з ключових складових УНГ, що забезпечує сервіси з обчислення та збереження даних для гріду в цілому.   Використання ресурсів СКІТ через хмарну підсистему, грід та безпосередньо у кластері допомагає науковими інститутам отримувати нові наукові результати та створювати цілу нові технології в широкому спектрі дисциплін: від генетики до лінгвістики. Проект присвячено розвитку ресурсного центру як інтегрованої складової частини національної грід-інфраструктури.

Архітектурні рішення, створені і впроваджені в процесі розробки ресурсного центру СКІТ, використовуються при побудові інших вітчизняних та закордонних обчислювальних систем.

І так далі про актуальність і важливість.

Результати проекту впроваджено у Ресурсному центрі Українського національного гріду. Розроблене системне та прикладне програмне забезпечення доступне для широкого загалу користувачів з НАН і МОН України. 
