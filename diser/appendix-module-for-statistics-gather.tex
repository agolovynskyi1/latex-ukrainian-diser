\chapter{РЕАЛІЗАЦІЯ МОДУЛЯ ЗБОРУ СТАТИСТИКИ}
\label{app-module-for-statistics-gather}

Для відображення даних з черги задач СКЗ та виконання розрахунку статистики роботи кластера розроблений Perl-модуль системи портала кластерних обчислень. Цей модуль  виконує обробку вхідного текстового лог-файлу менеджера ресурсів SLURM, а на виході формує оброблені дані черги задач, відфільтрованих за певном критерієм та обчислює статистичні показники, розглянуті вище. Модуль працює в автономному режимі (запуск за запитом з консолі) та доданий до функціоналу порталу кластерних обчислень.

Perl-модуль використовує файл конфігурації підсистеми hydra порталу кластерних обчислень, розміщену в /opt/hydra/etc/hydra.conf. Конфігураційні дані відображені на рис.~\ref{fig:slurm.perl}.

\begin{figure}[!hb]
\begin{lstlisting}
##Job-log section
#Path for job-log file or pipe command "cmd|"
JOB_LOG_FILE="ssh n1000 cat /var/log/slurm.job.log|"
#Identifie usernamemes from uid's in job-log, can by "yes" or "no",
LOG_USER_IDENT="yes"
\end{lstlisting}
  \caption{Конфігурація Perl-модуля статистики роботи кластера}
 \label{fig:slurm.perl}
\end{figure}

Параметр \textbf{JOB\_LOG\_FILE} вказує шлях до вхідного лог-файла SLURM або є конвеєрною командою, якщо лог-файл знаходиться на іншому сервері. Параметр \textbf{LOG\_USER\_IDENT} вказує ідентифікувати чи ні імена користувачів по їхнім  UID. Керівний демон SLURM кластера СКІТ  slurmctld функціонує на сервері, який з метою безпеки не забезпечує аутентифікацію користувачів через LDAP. У наслідок чого slurmctld не може по UID користувача, що запустив задачу, визначити його імені. Через це в лог-файлі черги задач фігурують тільки UID користувачів, а ім'я ідентифікується як Unknown. Саме для таких випадків і потрібна ця опція. Perl-модуль  напряму з’єднується з LDAP-сервером, та отримує від нього необхідну інформацію. Параметри підключення беруться з файла адміністративної конфігурації підсистеми hydra порталу кластерних обчислень, розміщену у \textbf{/opt/hydra/etc/hydra-adm.conf}. Параметри налаштування роботи з LDAP-сервером відображені на рис.~\ref{fig:slurm.perl.conf}.

\begin{figure}[!hb]
\begin{lstlisting}
# User database management system LDAP,PASSWD
USERDB=LDAP
# LDAP Parameters
LDAP_HOST=10.0.0.254
#LDAP_SEARCH_BASE="dc=cluster,dc=icyb"
LDAP_SEARCH_BASE="ou=People,dc=cluster,dc=icyb"
LDAP_ADM_DN="cn=ldapadm,dc=cluster,dc=icyb"
LDAP_ADM_PASSWD="password"
\end{lstlisting}
  \caption{Конфігурація Perl-модуля статистики роботи кластера для роботи з базою LDAP}
 \label{fig:slurm.perl.conf}
\end{figure}

Додатково модуль виконує обрахунок тривалості кожної задачі за часом її запуску та завершення. Цей параметр використовується у розрахунках статистики, а також виводится у форматі ''дні -- години : хвилини : секунди'' при відображенні інформації з черги задач. Модуль може відфільтровувати чергу задач за наступними параметрами:

\begin{itemize}
\item початкова та кінцева дата, в межах року, що задають період, за яким відображатиметься інформація про завершені задачі;
\item ім’я користувача (у профілях користувачів порталу кластерних обчислень завжди виводиметься інформація, відфільтрована цим критерієм);
\item статус завершення задачі.
\end{itemize}

Особливістю ведення журналу задач системи SLURM є відсутність значення року у датах запуску та завершення задач, тому необхідно розділяти лог-файли за роками.

Приклад результуючого виводу стану завершених задач, придатного для відображення в табличному вигляді системою порталу кластерних обчислень, зображений на рис.~\ref{fig:job.log}.

\begin{figure}[!hb]
\begin{lstlisting}
22988 solo(15026) run.ompi run.ompi COMPLETED scit3 20000 11/02-13:29:45
      11/02-13:29:47 0-0:0:2 n3137 1 8
22799 ru(5005) part2 part2 COMPLETED scit3 20000 10/27-13:30:09
      11/02-14:10:55 6-0:40:46 n[3106-3112,3143] 8 64
22990 yershov(5599) btree btree FAILED scit3 120 11/02-14:10:56
      11/02-14:15:34 0-0:4:38  n[3106-3112,3143] 8 64
22991 vgor(1111) gms gms COMPLETED scit3 20000 11/02-14:15:36
      11/02-14:38:43 0-0:23:7 n[3106-3109] 4 32
22987 engraver(1255) D78h26_searchField D78h26_searchField
      CANCELLED scit3 300 11/02-12:43:48 11/02-15:43:04 0-2:59:16 n3119 1 8
22996 engraver(1255) vortexD78h26 vortexD78h26 CANCELLED scit3 600
      11/02-15:43:16 11/02-16:01:55 0-0:18:39 n3119 1 8
\end{lstlisting}
  \caption{Розпарсений вивід стану завершених задач}
 \label{fig:job.log}
\end{figure}

Розрахована статистика відображається як для окремого користувача, так і в загальному по кластеру. Системним адміністраторам порталу доступне відображення статистики за всіма користувачами одночасно. Приклад відображення статистики використання кластера за січень  поточного року зображено на рис.~\ref{fig:job.stat}.

\begin{figure}[!hb]
\begin{lstlisting}
# ./ic_get_jobstat -s -b 01/01 -e 02/01
USER:JobCount,JobRatio(%),CD,CA,F,NF,TO,AvgJobTime(min),Load(TotalTime(min)*NCPU),LoadRatio(%)
Chobal: 30      1.35%   27      2       0       0       1       1552    1477135 27.41%
akn:    61      2.75%   50      5       6       0       0       5       2463    0.05%
alex:   124     5.58%   90      10      15      0       9       0       463     0.01%
belous: 43      1.94%   35      8       0       0       0       360     136676  2.54%
dept125:62      2.79%   27      9       26      0       0       0       3       0.00%
dept175:400     18.00%  193     145     62      0       0       9       25911   0.48%
glad:   7       0.32%   4       0       2       0       1       1       98      0.00%
ka:     55      2.48%   48      4       3       0       0       2       987     0.02%
kaa:    201     9.05%   188     7       6       0       0       24      742924  13.79%
karpman:13      0.59%   4       6       0       0       3       5       1064    0.02%
logvina:155     6.98%   82      19      54      0       0       0       4       0.00%
malyuga:1       0.05%   0       0       1       0       0       0       0       0.00%
pere:   5       0.23%   0       0       0       0       5       20000   800025  14.85%
koverov:8       0.36%   5       3       0       0       0       3379    149732  2.78%
root:   32      1.44%   28      0       3       1       0       0       3532    0.07%
ru:     24      1.08%   17      3       4       0       0       148     159688  2.96%
solo:   123     5.54%   70      44      9       0       0       168     1362447 25.28%
soro    175     7.88%   112     9       54      0       0       39      223762  4.15%
tikus:  27      1.22%   22      4       1       0       0       3       1164    0.02%
vgor:   72      3.24%   61      1       10      0       0       44      204214  3.79%
xand:   5       0.23%   4       0       1       0       0       3       83      0.00%
yakim:  596     26.82%  357     107     132     0       0       7       96320   1.79%
yrubin: 3       0.14%   3       0       0       0       0       1       34      0.00%
--------------------------------------------------------------------------
TOTAL: JobCount, CD,CA,F,NF,TO, AvgJobTime(min), Load(TotalTime(min)*NCPU)
TOTAL:  2222    1427    386     389     1       19      106     5388739
\end{lstlisting}
  \caption{Відображення статистики використання кластера}
 \label{fig:job.stat}
\end{figure}