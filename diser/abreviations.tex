\chapter{ПЕРЕЛІК УМОВНИХ ПОЗНАЧЕНЬ, СИМВОЛІВ, ОДИНИЦЬ, СКОРОЧЕНЬ І ТЕРМІНІВ}
% За алфавітом

\begin{description}
\item[AMD64/EM64T] --- 64-розрядна мікропроцесорна архітектура
\item[BIOS] --- базова система вводу/виводу
% \item[BlueGene] --- проект суперкомп’ютерної архітектури, розроблений фірмою IBM
\item [ccNUMA] (англ. cache coherent Non Uniform Memory Access) --- системи з неоднорідним доступом до пам'яті та когерентним кешем)
% \item[CELL] --- мікропроцесорна архітектура
% \item[Cloud computing] --- технологія обробки даних на базі Internet-сервісів
\item[CPU] --- центральний процесор
\item[DDR] --- подвійна швидкість передачі даних мережі Infiniband
% \item[Fat Tree] --- топологія мережевої комутації типу «Товсте дерево»
% \item[Gbps] --- гігабіт за секунду
% \item[GHz] --- гігагерц
\item[GPU] --- графічний процесор
% \item[HPC] (англ. High perfomance computing)--- високопродуктивні обчислення
\item[IBM] --- американська корпорація, виробник обчислювальної техніки
% \item[ICL DAP] --- мультипроцесорна система з масовим паралелізмом виробництва ICL
\item[Infiniband, Myrinet, Quadrix] --- високопродуктивні мережі передачі даних
\item[IP] (англ. Internet protocol) --- мережевий протокол
\item[IPMI] --- інтелектуальний інтерфейс управління платформою
\item[ISA] --- шина вводу/виводу IBM персональних комп'ютерів
\item[LDAP] --- протокол доступу до каталогів
\item[Linpack] --- бібліотека програм для тестування комп'ютерів
\item[Linux] --- UNIX- подібна операційна система
\item[Lustre] --- розподілена файлова система
\item[MPI] (англ. Message Passing Interface) --- інтерфейс обміну даними між обчислювальними процесами
\item[MPP] (англ. Massivlly Parallel Processing) --- паралельна обчислювальна система з масивом процесорів
% \item[NAT] --- механізм перетворення мережевих адрес
\item[NFS] --- мережева файлова система
\item[NUMA] (англ. Non Uniform Memory Access) --- системи з неоднорідним доступом до пам'яті
\item[OpenMP] --- мова та система паралельного програмування із застосуванням мов FORTRAN (77, 90 і 95), С та С++
\item[PAM] --- модулі аутентифікації операційної системи Linux
% \item[Perl] --- мова програмування високого рівня
\item[RAID] --- надлишковий масив незалежних дисків
\item[RAM] --- запам'ятовуючий пристрій з довільним доступом
% \item[RedHat] --- дистрибутив операційної системи Linux
\item[SLURM] --- менеджер ресурсів кластера
% \item[SSD] --- твердотільній дисковий накопичувач на базі флеш-пам’яті
% \item[U] --- одиниця вимірювання висоти обладнання в стійці
\item[UID] --- числовий ідентифікатор користувача операційної системи
% \item[UNIX] --- сімейство багатозадачних операційних систем
% \item[UPS] --- джерело безперебійного живлення
% \item[Web-портал] --- Web-сайт, що надає користувачам різноманітні інтерактивні сервіси
% \item[x86] --- мікропроцесорна архітектура
\item[БПЗ] --- базове програмне забезпечення
\item[ЕОМ] --- електронна обчислювальна машина
\item[ІК НАНУ] --- Інститут кібернетики ім. В.М.~Глушкова НАН України
\item[МОС] --- мультипроцесорна обчислювальна система
\item[ОС] --- операційна система
\item[ПЕ] --- процесорний елемент
\item[ПЗ] --- програмне забезпечення
% \item[СЗД] --- система збереження даних
\item[СКЗ] --- система керування задачами
\item[СКІТ] --- суперкомп'ютер інформаційних технологій
\item[суперкомп'ютер] --- обчислювальна система, яка має визначні характеристики продуктивності у порівнянні з комп'ютерами загального вжитку, у роботі терміни ``суперкомп'ютер'', ``обчислювальний кластер'', ``обчислювальний комплекс'' використовуються як синоніми
\item[гібридний вузол] --- обчислювальний вузол, в якому для обрахунків використовуються як універсальні процесори, так і спеціалізовані акселератори
\end{description}